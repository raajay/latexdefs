\RequirePackage{enumitem}
\RequirePackage{setspace}
\RequirePackage{xparse}


% A dense itemize environment
\newenvironment{denseitemize}{
\begin{itemize}[noitemsep, topsep=0pt, leftmargin=1.2em]
}{\end{itemize}}

% A dense enumerate environment
\newenvironment{denseenum}{
\begin{enumerate}[noitemsep, topsep=0pt, leftmargin=1.2em]
}{\end{enumerate}}

% A environment for minor headings
\providecommand{\tightcaption}{}%
\renewcommand{\tightcaption}[1]{%
  \begin{spacing}{0.5}
    \caption{#1}
  \end{spacing}
}%

% An environment for inline heading
\providecommand{\inlineheading}{}%
\renewcommand{\inlineheading}[1]{%
  \vspace{1\baselineskip plus 1\baselineskip minus 1\baselineskip} \noindent \textbf{#1:}
}%

% Similar to previous
\providecommand{\minorheading}{}%
\renewcommand{\minorheading}[1]{%
  \vspace{0.5\baselineskip}
  \noindent \textbf{#1}
  \vspace{0.5\baselineskip}
}%

% An environment for writing a custom legend using Tikz
\newenvironment{customlegend}[1][]{%
  \begingroup
  \csname pgfplots@init@cleared@structures\endcsname
  \pgfplotsset{#1}%
}{%
  \csname pgfplots@createlegend\endcsname
  \endgroup
}%

% A command to change the image in the custom legend.
\def\addlegendimage{\csname pgfplots@addlegendimage\endcsname}

% Use the \S symbol for referencing a section in a concise manner
\providecommand{\secref}{}
\renewcommand{\secref}[1]{Section~\ref{#1}}

% Same as above, just another macro name
\providecommand{\xref}{}
\renewcommand{\xref}[1]{Section~\ref{#1}}

% custom commands for reference
%\newcommand{\eqref}[1]{Equation~\ref{#1}}
\newcommand{\minisection}[1]{\smallskip\noindent{\bf #1.}}
\newcommand{\secsref}[2]{{\S\ref{#1} and \S\ref{#2}}}
\newcommand{\figref}[1]{Figure~\ref{#1}}
\newcommand{\figsref}[2]{{Figure~\ref{#1} and \ref{#2}}}
\newcommand{\tabref}[1]{Table~\ref{#1}}
\newcommand{\appref}[1]{{Appendix~\ref{#1}}}
\newcommand{\thmref}[1]{{Theorem~\ref{#1}}}
\newcommand{\lemref}[1]{{Lemma~\ref{#1}}}
\newcommand{\corref}[1]{{Corollary~\ref{#1}}}
